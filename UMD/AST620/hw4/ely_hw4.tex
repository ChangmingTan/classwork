\documentclass[12pt, 14paper]{report}
\usepackage{fancyhdr}
\pagestyle{fancy}
\lhead{ Justin Ely }
\chead{ ASTR620 - HW4}
\rhead{ Due 12/03/2013 }
\begin{document}

\section*{1) A}
The G-dwarf problem is the discrepancy between observations and models for the chemical evolution in galaxies.  Specifically, there are fewer metal-poor G-dwarfs observed than the simple models predict.

\section*{1) B}
One possible solution to the problem is to assume a non-zero initial metallicity to the system.  By assuming that the gas which formed the disk stars in the milky way was already pre-enriched, instead of assuming an initial metallicitiy of 0 as the simple models do, the models will predict fewer metal-poor G-dwarfs and will better align with observations.

Another possibility is to allow inflow into the system instead of assuming the system is completely isolated, as the closed box model does.  This inflow of gas would allow star-formation to continue over a prolonged period of time and continuously build up the metallicity of the stellar population.  Thus, the models would again predict fewer low-metallicty G-dwarfs.

\section*{1) C}
\begin{equation}
M_{\ast}(<Z) = M_{tot}[ 1 - exp(-\frac{ Z - Z(0) }{ y_Z} ) ]
\end{equation}

Using Equation 1 with no pre-enrichment, Z(0) = 0, the fraction of stars with metallicity $\le$.25$Z_{sun}$ is $\sim$.5 relative to the stars with average metallicity, as was done in lectures.  A pre-enrichment of Z(0)=.004 yields a values of $\sim$.15 which brings the predicted numbers of low-metal dwarfs closer to the observed value, but doesn't completely represent the observed fraction of .02.

\section*{2) A}
'Winding up' of spiral arms is the process by which differential rotation causes any cluster of gas/stars to become increasingly extended over time, thereby wrapping around the galaxy again and again.  This eventually causes a dissolution of the spirals as they are continuously spread out over multiple revolutions.  This is an issue because grand-design spirals are observed have large, long-lived arms that persist for longer than wind-up would allow.  This places limits on what kinds of spiral arms can be created by simple differential rotation and how long they can survive.

\section*{2) B}
A solution for grand-design spirals is to produce spiral arms as density perturbations instead of physical collections of stars undergoing differential rotation.  Instead of being a physical association of stars, the density perturbations move at a slower speed than the galactic rotation curve and stars are able to pass through.  The arms kick off star formation as they rotate, and the bright blue stars die out shortly after the arms have moved very far.  This creates the illusion of physical arms of young stars.

\section*{3}
The three largest of the local group galaxies are each very unique despite their close proximity.  M31 and the Milky way are the most similar in terms of over-all structure and appearance; both being large spirals and the two main components of the local group while M33 is smaller.  M31 and the MW both have large central bulges, while M33 has no bulge component.  M31 and MW have very similar velocity profiles, with similar components peaking around 250 km/s.  M33 however lacks a central bulge component, and has a much lower maximum velocity of only $\sim$ 120 km/s.  All three galaxies show star formation rates on the order of 1$M_{\odot}$ per year, but MW is the highest at $\sim$ 1$M_{\odot}$, M31 and M33 being .6$M_{\odot}$ and .45$M_{\odot}$ respectively.
\begin{center}
  \begin{tabular}{| l || c | c | c |}
    \hline
     & MW & M31 & M33 \\ \hline
     Max Rotation & 239km/s & 250km/s &120km/s \\ \hline
     Mass Dist. & buldge,disk,halo,DM & buldge,disk,halo,DM & disk,halo,DM \\ \hline
    Star Form. & 1$M_{\odot}$/yr & .6$M_{\odot}$/yr & .45$M_{\odot}$/yr \\ \hline
    BH & $\sim$4E6$M_{\odot}$ & $\sim$10E8$M_{\odot}$ & $\le$3E3$M_{\odot}$ \\
    \hline
  \end{tabular}
\end{center}

\section*{4}
"Spider Diagrams" are diagrams that gives the equations for lines of constant radial velocities as seen for a rotating galaxy inclined to the observer's line of sight.  They can show if the galaxies' velocity field inclination is warped or constant and if spiral arms or bars are present. 

\end{document}