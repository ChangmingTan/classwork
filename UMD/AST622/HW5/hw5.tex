\documentclass[a4paper,11pt]{article}
\usepackage{graphicx}

\begin{document}

\begin{flushright}

\vspace{1.1cm}

%Put ISR Title
{\bf\Huge Problem Set 5}

\rule{0.25\linewidth}{0.5pt}

\vspace{0.5cm}
%Put Authors
Justin Ely
\linebreak
\newline
%Put Author's affiliations
\footnotesize{AST622 University of Maryland College Park, MD\\}
\vspace{0.5cm}
% Date here below
26 April, 2012
\end{flushright}

\noindent\rule{\linewidth}{1.0pt}
\section*{1) Timescales of structure formation}
Structures of a given scale R form when:

\begin{equation}
\sigma^2(R) = \frac{1}{2\pi^2}\int P(k)W(K,R)K^2dk = 1
\end{equation}

where

\begin{eqnarray}
P = P_0(k) T^2(k) \\
\nonumber \\ 
T^2(k) = \frac{1}{1+\beta(\frac{k}{k_{eq}})^4} \\
\nonumber \\
P_0 = AK^n \\
\nonumber \\
W(R) = \Theta(R^{-1}-k) 
\end{eqnarray}

and $n=1$, $A = 5x10^{-5}$, and $B = 3x10^{-4}.$  The integral will be evaluated over the interval 0 to $1/R$, as the heavside step function will make the integral constant after $k = \frac{1}{R}$.

\begin{equation}
\sigma^2(R) = \frac{A}{2\pi^2} \int _o^{1/R} \frac{K^3}{1+\beta(\frac{k}{k_{eq}})^4}  dk =  \frac{A}{2\pi^2} \frac{\log(\frac{\beta}{k_{eq}^4} k^4 +1)}{\frac{4\beta}{k_{eq}}} 
\end{equation}

\begin{equation}
\sigma^2(R) = \frac{\log(\frac{\beta}{k_{eq}^4} k^4 +1)}{\frac{r\beta}{k_{eq}}}
\end{equation}

And from here I am not sure where to go.  I don't see how to get this in terms of a time.  It seems everything is in k, which I know can also be expressed in terms of R.  But I don't see anywhere a time could come in. 

\section*{2) E-Foldings of Inflation}
Finding the number of e-foldings required to solve the horizon problem is described in section 7.8.2 of the textbook.  From $\frac{\dot{a}}{a_0}^2 = H_0^2 [ \Omega_{o,w}(\frac{a_0}{a})^{1+3w} + (1-\Omega_{o,w})]$ and by requiring that the initial co-moving radius is much greater than than the co-moving radius in the current epoch, the number of e-foldings can be determined from eqn. 7.8.17 in the book as shown below.

\begin{equation}
N = \frac{60}{|1+3w|} [2.3 + \frac{1}{30}\ln(\frac{T_f}{T_p}) - \frac{1}{60}\ln(Z_{eq})]
\end{equation}


With $w=1$,$ Z_{eq} = 3800$, and $T_f/T_p = \Delta E = 10^{28}$, the number of e-foldings is then given by: 

\begin{equation}
N = \frac{60}{|4|} [2.3 + \frac{1}{30}\ln(10^{28}) - \frac{1}{60}\ln(3800)] = 65.83
\end{equation}

\end{document}
