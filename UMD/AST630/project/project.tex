\documentclass[a4paper,11pt]{article}
\usepackage{graphicx}

\begin{document}

\begin{flushright}

\vspace{1.1cm}

%Put ISR Title
{\bf\large Dusty White Dwarfs and the Late Stages of Planetary Systems}

\rule{0.25\linewidth}{0.5pt}

\vspace{0.1cm}
%Put Authors
Justin Ely \\
\vspace{0.1cm}
%Put Author's affiliations
\footnotesize{AST630 University of Maryland College Park, MD\\}
\vspace{0.1cm}
\end{flushright}

\noindent\rule{\linewidth}{1.0pt}
\section*{Abstract}
Observations of white dwarfs (WD) that show metal enrichment and dusty disks are thought to
be caused by planetesimals that survive main sequence evolution and are subsequently 
disrupted by the central star. 
Studies of the chemical abundances and 
dynamics of these observed enrichments suggest that the material is consistent with 
terrestrial bodies, which links the dusty disk to the dynamical evolution of the 
planetary system, though the precise physical mechanism is not well understood.  
The recent study by Debes et al. 2012 used numerical N-body simulations to 
demonstrate that the dusty disk around white dwarfs can be produced by 
interior mean motion resonances (IMMRs) between a giant planet and surviving planetesimals.  The IMMR 
perturbs the planetesimals from the asteroid belt into highly eccentric orbits 
that eventually cross close enough to the central star to be tidally disrupted. 
This method appears to be most efficient at the 2:1 IMMR, and can be shown
to provide material to the disk over timescales comparable to what is needed
to fit observations.

This study is important because it can help to tie down the eventual fate of 
solar systems similar to our own, as well as supplying information about the 
type of planetary systems that existed during the main sequence evolution of 
observed WDs.  
In particular, this method can be used to estimate the frequency and mass of 
asteroid belts as well as the characteristics of large surviving planets in WD
systems.

The Debes et al. article explored the ability of IMMR to produce the observed WD 
characteristics starting with initial conditions found in our solar system.  To fully understand the broad-scale applicability
of this phenomenon as a whole requires understanding how it would behave in other solar systems beyond our own.  
Following the methods used by Debes et al., I wil use the MERCURY n-body simulation code to repeat the mass accretion
simulation over a wider parameter space.  By varying characteristics such as the mass of the central star, mass and semi-major axis of the large planet, and initial distribution of planitesimals, we will understand more about where IMMR is efficient at supplying mass to the central WD and to what type of systems we would expect this to be a significant factor. 

\section*{References}
Debes, J. H. and Walsh, K. J., Stark, C., 2012, ApJ, 747, 148 \\

\end{document}
