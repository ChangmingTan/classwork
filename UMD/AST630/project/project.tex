\documentclass[a4paper,11pt]{article}
\usepackage{graphicx}

\begin{document}

\begin{flushright}

\vspace{1.1cm}

%Put ISR Title
{\bf\large Dusty White Dwarfs and the Late Stages of Planetary Systems}

\rule{0.25\linewidth}{0.5pt}

\vspace{0.1cm}
%Put Authors
Justin Ely \\
\vspace{0.1cm}
%Put Author's affiliations
\footnotesize{AST630 University of Maryland College Park, MD\\}
\vspace{0.1cm}
% Date here below
13 October, 2014
\end{flushright}

\noindent\rule{\linewidth}{1.0pt}
\section*{Abstract}
Exoplanet discovery has exploded in recent decades.  After detections of 
hundreds of extrasolar planets around many different types of stars and stellar
remnants, we can conclude that planetary systems exist in abundance in 
the galaxy in various states of evolution.  Arguments for the existense of planetary systems surviving main sequence 
evolution and existing around white dwarfs (WD) have been 
supported by observations of WD stars with heavy metal enrichment and infra-red
excess indicative of dust.  Further studies of the chemical abundances  and 
dynamics of these
enrichments concluded that the material is reminiscent of terrestrial planets 
which had been tidally disrupted. 

The recent study by Debes et al. 2012 used numerical N-body simulations to 
demonstrate that the dusty disk around white dwarfs can be produced by 
interior mean motion resonances (IMMRs) between surviving planetesimals and
 a giant planet.  The IMMR 
perturbs the planitesimals from the asteroid belt into highly eccentric orbits 
that eventually cross close enough to the central star to be tidally disrupted. 
This method appears to be most efficient at the 2:1 resonance, and can be shown
to provide material to the disk over timescales comparable to what is needed
to fit observations.

This study is important because it can help to tie down the eventual fate of 
solar systems similar to our own, as well as supplying information about the 
type of planetary systems that existed during the main sequence evolution of 
observed WDs.  
In particular, this method can be used to estimate the frequency and mass of 
asteroid belts as well as the characteristics of large surviving planets in WD
systems.

The question of whether or not the disruption of the planitesimal eventually 
forms a disk that matches the observed properties of dusty white dwarfs was left
 out of this study as too computationally intensive.  I will expand on this 
work to simulate the end evolution of the dusty disk.


\end{document}
