\documentclass[a4paper,11pt]{article}
\usepackage{graphicx}
\usepackage{enumerate}

\begin{document}

\begin{flushright}

\vspace{1.1cm}

{\bf\Huge Problem Set 1}

\rule{0.25\linewidth}{0.5pt}

\vspace{0.5cm}
%Put Authors
Justin Ely
\linebreak
\newline
%Put Author's affiliations
\footnotesize{605.411 Foundations of Computer Architecture \\}
\vspace{0.5cm}
% Date here below
08 September, 2016
\end{flushright}

\noindent\rule{\linewidth}{1.0pt}

%%%%%%%%%%%%%%%%%%%%%%%%%%%%%%%%%%%%%%%%%%%%%%%%%%%%%%%%%%

\section*{1a)}
The clock rate is the inverse of the clock cycle.  Thus, the clock rate for this implementation is $\frac{1}{t_{cycle}} = \frac{1}{250e-12} = 4GHz$.

\section*{1b)}
The total number of clock cycles consumed is calculated by the total number of instructions multipled by the weighted sum of the cycles per instruction.\\

\noindent $N_{cycles} = 20e6 (.1 \times 12 + .9 \times 5) = 1.14E8 $ 

\section*{1c)}
Making the divide instruction twice as fast means cutting the number of cycles in half from 12 to 6. \\

\noindent $N_{cycles\_new} = 20e6 (.1 \times 6 + .9 \times 5) = 1.02E8 $  \\

Thus, the speedup is the ratio between the new and old number of cycles required. \\

\noindent $speedup = \frac{N_{cycles}}{N_{cycles\_new}} = 1.12$ or $12\%$. 

%%%%%%%%%%%%%%%%%%%%%%%%%%%%%%%%%%%%%%%%%%%%%%%%%%%%%%%%%%

\section*{2a)} 
$CPI = time \times \frac{rate}{n_{inst}} = 1.1 \times \frac{1e-9}{1e-9} = 1.1$

\section*{2b)} 
$T2 = 1.1 \times \frac{6e8}{\frac{1}{1e-9}} = .66$

\noindent Thus the speedup is: $\frac{T1}{T2} = \frac{1.1}{.66} = 1.67$ or $67\%$. 

%%%%%%%%%%%%%%%%%%%%%%%%%%%%%%%%%%%%%%%%%%%%%%%%%%%%%%%%%%

\section*{3a)}
Ignoring constant terms, which cancel, and taking the fraction of multiply instructions and other instructions to be $x$ and $y$ respectively,

\begin{eqnarray}
t_1 &=& 4x + 4y \\ 
2.85 &=& 4x + 4y 
\end{eqnarray}

\begin{eqnarray}
t_2 &=& 3x + 4y \\
2.28 &=& 3x + 4y
\end{eqnarray}

Re-ordering (3) and (4), and substituting variables gives us:

\begin{eqnarray}
2.85 - 4x &=& 2.28 - 3x \\
2.85 &=& 2.28 + x \\
.57 &=& x
\end{eqnarray}

The multiply instruction accounts for 57\% of the executed instructions.

\section*{3b)}
Since speedup is $\frac{t1}{t2}$, constant-terms can again be ignored.

\begin{eqnarray}
t_1 &=& (12 \times .22 + 4 \times .78) \\
t_2 &=& (3 \times .22 + 4 \times .78)
\end{eqnarray}

Combining (8) and (9) gives $\frac{t_1}{t_2} = 1.52$, which is the speedup factor.

%%%%%%%%%%%%%%%%%%%%%%%%%%%%%%%%%%%%%%%%%%%%%%%%%%%%%%%%%%

\section*{4a)}
The total clock cycles for each mode will be the weighted sum of the CPI * instruction count.

\begin{itemize}
	\item P1: $1E6 (1 \times .1 + 2 \times .2 + 2 \times .5 + 3 \times .2) = 2.1e6$ cycles.
	\item P1: $1E6 (2 \times .1 + 2 \times .2 + 2 \times .5 + 2 \times .2) = 2.0e6$ cycles.
\end{itemize}

\section*{4b)}
The total running time will be the number of cycles times the clock cycle time ($\frac{1}{rate}$).

\begin{itemize}
	\item P1: $2.1e6 \times \frac{1}{2.5e9} = .00084$ seconds.
	\item P2: $2.0e6 \times \frac{1}{3e9} = .00067$ seconds.
\end{itemize}


\section*{4c)}
The CPI for each implementation is the weighted sum of the CPIs of the individual instruction classes.

\begin{itemize}
	\item P1: $1 \times .1 + 2 \times .2 + 2 \times .5 + 3 \times .2 = 2.1$ CPI.
	\item P1: $2 \times .1 + 2 \times .2 + 2 \times .5 + 2 \times .2 = 2.0$ CPI.
\end{itemize}

%%%%%%%%%%%%%%%%%%%%%%%%%%%%%%%%%%%%%%%%%%%%%%%%%%%%%%%%%%

\section*{5a)}
The initial running time is 250s, which consists of 70s on FP, 85s on L/S, 40s on branching, and 55s on other instructions.

If FP instructions are reduced by 20\%, the FP instructions will then take $70 * .8 = 56$.  Thus, the new total time will be $56 + 85 + 40 + 55 = 236$, which is a reduction of 14 seconds from the original.

\section*{5b)}

To reduce the total time by 20\%, we would need to cut off $250 \times .2 = 50$ seconds.  Since branching instructions currently only take 40 seconds, even if they took 0 seconds the goal time could not be reached.

%%%%%%%%%%%%%%%%%%%%%%%%%%%%%%%%%%%%%%%%%%%%%%%%%%%%%%%%%%



\end{document}
