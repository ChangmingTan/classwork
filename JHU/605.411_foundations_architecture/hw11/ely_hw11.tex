\documentclass[a4paper,11pt]{article}
\usepackage{graphicx}
\usepackage{enumerate}
\usepackage[usenames, dvipsnames]{color}

\begin{document}

\begin{flushright}

\vspace{1.1cm}

{\bf\Huge Problem Set 11}

\rule{0.25\linewidth}{0.5pt}

\vspace{0.5cm}
%Put Authors
Justin Ely
\linebreak
\newline
%Put Author's affiliations
\footnotesize{605.411 Foundations of Computer Architecture \\}
\vspace{0.5cm}
% Date here below
22 November, 2016
\end{flushright}

\noindent\rule{\linewidth}{1.0pt}

%%%%%%%%%%%%%%%%%%%%%%%%%%%%%%%%%%%%%%%%%%%%%%%%%%%%%%%%%%

\section*{1a)}
$P_{0-3}` = B0 \oplus B1 \oplus B2` \oplus B3$

\section*{1b)}
\begin{eqnarray}
P_{0-3}` = B0 \oplus B1 \oplus B2` \oplus B3 \\
P_{0-3}` \oplus B2` = B0 \oplus B1 \oplus B3 
\end{eqnarray}

\begin{eqnarray}
P_{0-3} = B0 \oplus B1 \oplus B2  \oplus B3 \\
P_{0-3} \oplus B2 = B0 \oplus B1 \oplus B3 
\end{eqnarray}

\begin{eqnarray}
P_{0-3} \oplus B2 = B0 \oplus B1 \oplus B3  = P_{0-3}` \oplus B2` \\
P_{0-3} \oplus B2  = P_{0-3}` \oplus B2` \\
P_{0-3} \oplus B2 \oplus B2` = P_{0-3}`  
\end{eqnarray}

%%%%%%%%%%%%%%%%%%%%%%%%%%%%%%%%%%%%%%%%%%%%%%%%%%%%%%%%%%

\section*{2a)} 
This system is a RAID5: striping with single parity shared across disks.

\section*{2b)} 
To write new content to block 8, we also need to write a new parity block, so W=2.  Writing a new parity block requires reading the 4 data blocks, 
so R = 4.  Computing the new parity is $P_{8-11} = B8 \, xor  \, B9  \, xor  \, B10  \, xor  \, B11 $ so N = 3.

%%%%%%%%%%%%%%%%%%%%%%%%%%%%%%%%%%%%%%%%%%%%%%%%%%%%%%%%%%

\section*{3a)}
RAID 4 strips data with a single parity disk, so $N_{disks} = 4T + 1$.

\section*{3b)}
RAID 5 also strips data with the equivalent of 1 parity disk striped across the volumes, so $N_{disks} = 4T + 1$.

\section*{3c)}
RAID 6 also strips data with the equivalent of 2 parity disk striped across the volumes, so $N_{disks} = 4T + 2$.

\section*{3d)}
RAID 1 mirrors all the data with no parity, so $N_{disks} = 2 \times 4T = 8T$.

%%%%%%%%%%%%%%%%%%%%%%%%%%%%%%%%%%%%%%%%%%%%%%%%%%%%%%%%%%

\section*{4a)}
N = 2: 

\section*{4b)}
N = 2.

\section*{4c)}
If disks 2 and 6 fail, only the RAID10 system will operate.  In the RIAD01 system, strips 1, 5, 9, 13 would no longer
be accessible.

%%%%%%%%%%%%%%%%%%%%%%%%%%%%%%%%%%%%%%%%%%%%%%%%%%%%%%%%%%

\section*{5a)}
D, A, B, C

\section*{5b)}
Device B would then not be able to have requests granted.

%%%%%%%%%%%%%%%%%%%%%%%%%%%%%%%%%%%%%%%%%%%%%%%%%%%%%%%%%%


\end{document}
