\documentclass[a4paper,11pt]{article}
\usepackage{graphicx}
\usepackage{enumerate}
\usepackage[usenames, dvipsnames]{color}

\begin{document}

\begin{flushright}

\vspace{1.1cm}

{\bf\Huge Problem Set 9}

\rule{0.25\linewidth}{0.5pt}

\vspace{0.5cm}
%Put Authors
Justin Ely
\linebreak
\newline
%Put Author's affiliations
\footnotesize{605.411 Foundations of Computer Architecture \\}
\vspace{0.5cm}
% Date here below
08 November, 2016
\end{flushright}

\noindent\rule{\linewidth}{1.0pt}

%%%%%%%%%%%%%%%%%%%%%%%%%%%%%%%%%%%%%%%%%%%%%%%%%%%%%%%%%%

\section*{1a)}
For direct mapping, $log_2(128) = 7$ is the number of bits in the offset field and $log_2(131072) = 17$ is the number of
bits in the line field.  \\

\begin{tabular}{| l | c | c | c |}
  \hline	
      & Line & Offset  \\  \hline \hline
      1000 0010 & 0101 0001 1100 1010 1 & 100 1100 \\ \hline
\end{tabular} \\

\section*{1b)}
For a 2-way set associative, there will be 2 lines per set.  For this, the offset stays at 7 bits, and the set becomes $log_2(\frac{131072}{2}) = 16$.

\begin{tabular}{| l | c | c | c |}
  \hline	
      & Set & Offset  \\  \hline \hline
      1000 0010  0 & 101 0001 1100 1010 1 & 100 1100 \\ \hline
\end{tabular} \\

\section*{1c)}
For 4-way set associative, there will be 4 lines per set.  With a 64-bit line, the offset will be $log_2(64) = 6$.  The set will be
$log_2(\frac{131072}{4}) = 15$. 

\begin{tabular}{| l | c | c | c |}
  \hline	
      & Set & Offset  \\  \hline \hline
      1000 0010  010 & 1 0001 1100 1010 11 & 00 1100 \\ \hline
\end{tabular} \\

\section*{1d)}
A 1024-byte line size results in an offset which is $log_2(1024) = 10$ bits.  For fully associative, the rest of the address becomes the tag:

\begin{tabular}{| l | c | c | c |}
  \hline	
      Tag & Offset  \\  \hline \hline
      1000 0010  0101 0001 1100 10 & 10 1100 1100 \\ \hline
\end{tabular} \\


%%%%%%%%%%%%%%%%%%%%%%%%%%%%%%%%%%%%%%%%%%%%%%%%%%%%%%%%%%

\section*{2a)} 
According to the table in lecture: if the bits are currently 001, a miss will result in replacing way 2.

\section*{2b)} 
Startin with 001, a hit on way 2 will cause:

\begin{itemize}
  \item Bit 2 to be set to 1
  \item Bit 1 to remain unchanged at 0
  \item Bit 0 to be set to 0
\end{itemize}

\noindent Which will result in bits 100.


%%%%%%%%%%%%%%%%%%%%%%%%%%%%%%%%%%%%%%%%%%%%%%%%%%%%%%%%%%

\section*{3a)}
$ Ave = .9 \times .5ns + .1 (.5 \times (.5ns + 50ns)) = 5.5 ns$.
 
\section*{3b)}
$ Ave = .9 \times .5ns + .1 (.6 \times (.5ns + 5ns) + .4 \times (.5ns + 5ns + 50ns)) = 3 ns$.

%%%%%%%%%%%%%%%%%%%%%%%%%%%%%%%%%%%%%%%%%%%%%%%%%%%%%%%%%%

\section*{4a)}
\begin{eqnarray}
10 &=& h \times 8 + (1-h) \times 60 \\
10 &=& 8 h + 60 - 60 h \\
-50 &=& 8 h - 60 h \\
-50 &=& -52 h \\
.9625 &=& h
\end{eqnarray}

\section*{4b)}
\begin{eqnarray}
10 &=& h \times 8 + (1-h) \times (60 + 8) \\
10 &=& 8 h + 68 - 68 h \\
-58 &=& 8 h - 68 h \\
-58 &=& -60 h \\
.9666 &=& h
\end{eqnarray}

%%%%%%%%%%%%%%%%%%%%%%%%%%%%%%%%%%%%%%%%%%%%%%%%%%%%%%%%%%

\section*{4)}
c) access time/address space

%%%%%%%%%%%%%%%%%%%%%%%%%%%%%%%%%%%%%%%%%%%%%%%%%%%%%%%%%%

\section*{5a)}
The virtual address will be 4 bits for the page number and 128 bits (64 bytes) for the offset, which totals 132 bits.

\section*{5b)}
The physical address will be 3 bites for the frame number and 128 bits (64 bytes) for the offset, which totals 131 bits.

\section*{5c)}
{\it The value 0x97 doesn't seem like enough bits to give an address for this problem, so i may not be completely understanding this, however:}

The leftmost 4 bits (1001) will be used to 
index the page table.  If the frame at that index does exist, then the bits (001) will be used to find the correct frame.  


%%%%%%%%%%%%%%%%%%%%%%%%%%%%%%%%%%%%%%%%%%%%%%%%%%%%%%%%%%

\section*{6a)}
$ Ave = .9 \times .6ns + .1 (.6 + 70ns) = 7.6 ns$.

\section*{6b)}
$ Ave = .9 \times .6ns + .1 (.1 \times (.6 + 6) + .9 \times (.6 + 6 + 70)) = 7.5 ns$.

%%%%%%%%%%%%%%%%%%%%%%%%%%%%%%%%%%%%%%%%%%%%%%%%%%%%%%%%%%

\end{document}
