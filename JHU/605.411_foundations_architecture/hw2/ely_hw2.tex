\documentclass[a4paper,11pt]{article}
\usepackage{graphicx}
\usepackage{enumerate}

\begin{document}

\begin{flushright}

\vspace{1.1cm}

{\bf\Huge Problem Set 2}

\rule{0.25\linewidth}{0.5pt}

\vspace{0.5cm}
%Put Authors
Justin Ely
\linebreak
\newline
%Put Author's affiliations
\footnotesize{605.411 Foundations of Computer Architecture \\}
\vspace{0.5cm}
% Date here below
13 September, 2016
\end{flushright}

\noindent\rule{\linewidth}{1.0pt}

%%%%%%%%%%%%%%%%%%%%%%%%%%%%%%%%%%%%%%%%%%%%%%%%%%%%%%%%%%

\section*{1)}
\begin{tabular}{| l | c | c | c | c | c | c | c | c |}
  \hline	
  Hex & 7 & 1 & 5 & 4 & 0 & 0 & 0 & 0 \\  \hline  		
  Bin & 0111 & 0001 & 0101 & 0100 & 0000 & 0000 & 0000 & 0000 \\ \hline
\end{tabular} \\

\noindent  In IEEE 754 single precision, this binary value represents 1.05E30.

%%%%%%%%%%%%%%%%%%%%%%%%%%%%%%%%%%%%%%%%%%%%%%%%%%%%%%%%%%

\section*{2)} 
Using the unsigned representation of 12 = 00001100:

\section*{a)}
The excess-128 representation gives the signed value as 128 higher than the negative value.  Thus -12 = 128-12 = 116 = 01110100 or 0x74.

\section*{b)} 
1's complement representation of -12 can be found simply by taking the complement of every bit: 11110011 or 0xf3.

\section*{c)} 
The sign and magnitude representation simply uses the first bit as the sign, and the remaining 7 bits as the magnitude: 10001100 or 0x8c.

\section*{d)} 
Two's complement is found by taking the complement of every bit and then adding 1.  The complement being 11110011, with 1 added: -12 = 11110100 or 0xf4.

%%%%%%%%%%%%%%%%%%%%%%%%%%%%%%%%%%%%%%%%%%%%%%%%%%%%%%%%%%

\section*{3)}
\begin{tabular}{| l | c | c | c | c | c | c | c | c |}
  \hline	
   Sign & Expoent & Fraction \\  \hline  		
   0 & 10000100 & 11111010000000000000000  \\ \hline
\end{tabular} \\

%%%%%%%%%%%%%%%%%%%%%%%%%%%%%%%%%%%%%%%%%%%%%%%%%%%%%%%%%%

\section*{4)}
\begin{tabular}{| l | c | c | c | c | c | c | c | c |}
  \hline	
   Sign & Exponent & Fraction \\  \hline  		
   0 & 10000100000 & 1111101000000000000000000000000000000000000000000000  \\ \hline
\end{tabular} \\

%%%%%%%%%%%%%%%%%%%%%%%%%%%%%%%%%%%%%%%%%%%%%%%%%%%%%%%%%%

\section*{5)}
\begin{tabular}{| c | c | c | c | c | c |}
  \hline	
  Iteration & Step & Multiplier & Multiplicand & Product  \\  \hline  		
  0 & Init & 01100010 & 0000000000010010 & 0000000000000000   \\ \hline
  0 & 1: 0: nothing & 01100010 & 0000000000010010 & 0000000000000000   \\ \hline
  0 & 2: shift m-and left & 01100010 & 0000000000100100 & 0000000000000000   \\ \hline
  0 & 3: shift m-er right & 00110001 & 0000000000100100 & 0000000000000000   \\ \hline
  1 & 1: 1: add & 00110001 & 0000000000100100 & 0000000000100100   \\ \hline
  1 & 2: shift m-and left & 00110001 & 0000000001001000 & 0000000000100100   \\ \hline
  1 & 3: shift m-er right & 00011000 & 0000000001001000 & 0000000000100100   \\ \hline
  2 & 1: 0: nothing & 00011000 & 0000000001001000 & 0000000000100100   \\ \hline
  2 & 2: shift m-and left & 00011000 & 0000000010010000 & 0000000000100100   \\ \hline
  2 & 3: shift m-er right & 00001100 & 0000000010010000 & 0000000000100100   \\ \hline
  3 & 1: 0: nothing & 00001100 & 0000000010010000 & 0000000000100100   \\ \hline
  3 & 2: shift m-and left & 00001100 & 0000000100100000 & 0000000000100100   \\ \hline
  3 & 3: shift m-er right & 00000110 & 0000000100100000 & 0000000000100100   \\ \hline
  4 & 1: 0: nothing & 00000110 & 0000000100100000 & 0000000000100100   \\ \hline
  4 & 2: shift m-and left & 00000110 & 0000001001000000 & 0000000000100100   \\ \hline
  4 & 3: shift m-er right & 00000011 & 0000001001000000 & 0000000000100100   \\ \hline
  5 & 1: 1: add & 00000011 & 0000001001000000 & 0000001001100100   \\ \hline
  5 & 2: shift m-and left & 00000011 & 0000010010000000 & 0000001001100100   \\ \hline
  5 & 3: shift m-er right & 00000001 & 0000010010000000 & 0000001001100100   \\ \hline
  6 & 1: 1: add & 00000001 & 0000010010000000 & 0000011011100100   \\ \hline
  6 & 2: shift m-and left & 00000001 & 0000100100000000 & 0000011011100100   \\ \hline
  6 & 3: shift m-er right & 00000000 & 0000100100000000 & 0000011011100100   \\ \hline
  7 & 1: 0: nothing & 00000000 & 0000100100000000 & 0000011011100100   \\ \hline
  7 & 2: shift m-and left & 00000000 & 0001001000000000 & 0000011011100100   \\ \hline
  7 & 3: shift m-er right & 00000000 & 0001001000000000 & 0000011011100100   \\ \hline
  8 & 1: 0: nothing & 00000000 & 0001001000000000 & 0000011011100100   \\ \hline
  8 & 2: shift m-and left & 00000000 & 0010010000000000 & 0000011011100100   \\ \hline
  8 & 3: shift m-er right & 00000000 & 0010010000000000 & 0000011011100100   \\ \hline
  
\end{tabular} \\

The value left in the Product register evaluates to 1764, which is equal to the product of 0x62 (98) and 0x12 (18).


%%%%%%%%%%%%%%%%%%%%%%%%%%%%%%%%%%%%%%%%%%%%%%%%%%%%%%%%%%

\section*{6)}

\begin{tabular}{| c | c | c | c | c | c |}
  \hline	
  op code & source register & second source & dest. register & shift amount & function \\  \hline  		
  000000 & 01000 & 01001 & 10001 & 00000 & 100000  \\ \hline
  0 & 8 & 9 & 17 & 0 & 32 \\ \hline
  R-type & \$t0 & \$t1 & \$s1 & 0 & add \\\hline
\end{tabular} \\

\noindent  In HEX, this translates to:

\noindent \begin{tabular}{| l | c | c | c | c | c | c | c | c |}
  \hline	
  Bin & 0000 & 0001 & 0000 & 1001 & 1000 & 1000 & 0010 & 0000 \\ \hline
  Hex & 0 & 1 & 0 & 9 & 8 & 8 & 2 & 0 \\  \hline  		
\end{tabular} \\

%%%%%%%%%%%%%%%%%%%%%%%%%%%%%%%%%%%%%%%%%%%%%%%%%%%%%%%%%%

\section*{7)}
\begin{tabular}{| l | c | c | c | c | c | c | c | c |}
  \hline	
  Hex & 1 & 2 & 0 & F & 0 & 0 & 0 & 8 \\  \hline  		
  Bin & 0001 & 0010 & 0000 & 1111 & 0000 & 0000 & 0000 & 1000 \\ \hline
\end{tabular} \\

\noindent The first 6 bits signify an opcode of 4, which makes this an I-type.  \\

\noindent \begin{tabular}{| l | c | c | c | c | c | c | c | c |}
  \hline	
  opcode & rs & rt & immediate  \\  \hline  		
  000100 & 10000 & 01111 & 0000000000001000  \\ \hline
  beq & \$s0 & \$t7 & 8  \\ \hline
\end{tabular} \\

\noindent This translates into a MIPS command of "beq \$t7, \$s0, 8".

%%%%%%%%%%%%%%%%%%%%%%%%%%%%%%%%%%%%%%%%%%%%%%%%%%%%%%%%%%

\end{document}
